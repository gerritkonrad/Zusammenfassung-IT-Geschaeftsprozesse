\subsection{Übung}

\subsubsection*{Aufgabe (Effektiv vs. Effizient)}
\paragraph*{Aufgabe}
    Bewerten Sie, ob der nachfolgend beschriebene Prozess (aus dem persönlichen Umfeld) effektiv und effizient ist. Beachten Sie bei der Bewertung insb. die nachstehende Beobachtung. 

    Prozess: Lebensmittel einkaufen 
    \begin{itemize}
        \item Schritt 1: Einkaufsliste (Lebensmittel) schreiben 
        \item Schritt 2: Lebensmittelgeschäft auswählen
        \item Schritt 3: Zum Lebensmittelgeschäft fahren
        \item Schritt 4: Produkte gemäß Einkaufsliste auswählen 
        \item Schritt 5: Bezahlen 
        \item Schritt 6: Produkte verladen
        \item Schritt 7: Produkte nach Hause transportieren und in Schränke einräumen 
    \end{itemize}

    Beobachtung: Im Schritt 4 besuchen Sie die einzelnen Abteilungen (z.B. Obst, Reinigungsmittel) des Lebensmittelgeschäfts häufiger, da immer wieder noch ein Produkt aus der jeweiligen Abteilung fehlt. 

\paragraph*{Lösung}
    Nein, da einzelne Abteilungen immer mehrfach besucht werden, das wiederum braucht Zeit. Um das zu beheben sollte man die Lebensmittel nach Abteilung sortieren.
    Man sollte außerdem noch Preise pro Lebensmittelgeschäft vergleichen und das günstigste Lebensmittelgeschäft raussuchen.

\subsubsection*{Aufgabe (Rollen im Geschäftsprozessmanagement)}
\paragraph*{Aufgabe}
    \begin{enumerate}[label=\alph*)]
        \item Beschreiben Sie die Aufgaben, die der Rolle des „Prozessverantwortlichen“ im Rahmen des Geschäftsprozessmanagement zukommen.
        \item Welche weiteren Rollen sind für das Geschäftsprozessmanagement relevant?
    \end{enumerate}
   
\paragraph*{Lösung}
    \begin{enumerate}[label=\alph*)]
        \item Verantwortlich für die Ausführung und Anpassung der Prozesse
        \item Geschäftsführung, Prozessteilnehme, Systemanalytiker, Anwendungsentwickler
    \end{enumerate}

\subsubsection*{Aufgabe (Prozessverbesserung)}
\paragraph*{Aufgabe}
    Wie könnte das in Aufgabe 2.1 beobachtete Problem, das zur Verschwendung von Zeitressourcen führt, durch eine bessere Strukturierung der Einkaufsliste behoben werden?
   
\paragraph*{Lösung}
    In dem man die Lebensmittel auf der Einkaufsliste nach Abteilungen sortiert

\subsubsection*{Aufgabe (Sichten auf Prozesse)}
\paragraph*{Aufgabe}
    \begin{enumerate}[label=\alph*)]
        \item Welche Sichten werden bei der Darstellung von Prozessen unterschieden? Beschreiben Sie die genannten Sichten kurz.
        \item Beschreiben Sie im Kontext der Datensicht, welche Informationen eine Einkaufsliste (vgl. Aufgabe 2.1) enthalten sollte und wie diese Informationen strukturiert sein sollten, um das in Aufgabe 2.1 identifizierte Problem zu beseitigen.
        \item Welche Informationen müssen zur Erstellung einer in Teilaufgabe b) beschriebenen Einkaufsliste im Vorfeld vorliegen?
    \end{enumerate}
   
\paragraph*{Lösung}
    \begin{enumerate}[label=\alph*)]
        \item 
    \end{enumerate}


\subsubsection*{Aufgabe (Lebenszyklus des Geschäftsprozessmanagements)}
\paragraph*{Aufgabe}
    \begin{enumerate}[label=\alph*)]
        \item Welche Phasen werden im Lebenszyklus des Geschäftsprozessmanagements unterschieden?
        \item Zu welcher Phase des Lebenszyklus sind die in Aufgabe 2.4b durchgeführten Schritte zuzuordnen?
    \end{enumerate}
   
\paragraph*{Lösung}
    \begin{enumerate}[label=\alph*)]
        \item 
        \item 
    \end{enumerate}


\subsubsection*{Aufgabe (Prozessidentifikation)}
\paragraph*{Aufgabe}
    Identifizieren Sie die Prozesse die in einem typischen Prüfungsamt (z.B. im SSB der FH SWF) ausgeführt werden bzw. an denen ein typisches Prüfungsamt beteiligt ist und stellen Sie diese in Form einer selbst gewählten Prozesslandkarte dar.
   
\paragraph*{Lösung}


\subsubsection*{Aufgabe (Dimensionen zur Prozessverbesserung)}
\paragraph*{Aufgabe}
    \begin{itemize}
        \item Aktivität 1 (Warenwirtschaftssystem): Druckt Pickliste zum Kundenauftrag aus.
        \item Aktivität 2 (Mitarbeiter 1 - Hilfsmittel Einkaufskorb): Stellt die Waren manuell gemäß der Pickliste zusammen und bringt diese in den Kommissionierbereich.
        \item Aktivität 3 (Mitarbeiter 2): Prüft die Vollständigkeit der vom Mitarbeiter 1 kommissionierten Produkte nach Art und Anzahl.
        \item Aktivität 4 (Mitarbeiter 2): Verpackt die kommissionierten Waren und klebt das Versandetikett (Lieferadresse, Route) auf die Verpackung.
        \item Aktivität 5 (Mitarbeiter 3): Bringt das Paket aus dem Kommissionierbereich in den Warenausgangsbereich.
        \item Aktivität 6 (Mitarbeiter 4): Der Mitarbeiter 4 entnimmt das Paket aus dem Warenausgangsbereich und liefert es mit dem Lieferwagen an die auf dem Versandetikett hinterlegte Adresse
    \end{itemize}

    Versetzen Sie sich in folgende Situation: Die Kunden beschweren sich in letzter Zeit zunehmend darüber, dass sie abgelaufene Lebensmittel erhalten.

    Bearbeiten Sie die folgenden Aufgaben:
    \begin{enumerate}[label=\alph*)]
        \item Auf welche Dimension der Geschäftsprozessverbesserung bezieht sich die Beschwerde der Kunden?
        \item Entwickeln Sie einen (möglichst einfachen) Vorschlag zur Anpassung des Prozesses, um die Belieferung von Kunden mit abgelaufenen Lebensmitteln zu vermeiden oder zu reduzieren.
        \item Verdeutlichen Sie die Bedeutung des „Teufelsviereck des Geschäftsprozessmanagements“ an Ihrem Beispiel.
    \end{enumerate}
\paragraph*{Lösung}


\subsubsection*{Aufgabe (Redesign Heuristiken)}
\paragraph*{Aufgabe}
    Lesen Sie die Erläuterung zu den Heuristiken im Lehrbuch auf den Seiten 118 - 120 und machen Sie einen Vorschlag (inkl. Begründung!) welche Heuristik Sie in den folgenden Situationen anwenden würden, um die beschriebenen Probleme zu lösen bzw. zu verringern:
    
    \begin{enumerate}[label=\alph*)]
        \item Ein Produktentwicklungsprozess (im Individualmaschinenbau) wird durch häufiges Nachfragen der Entwickler beim Kunden stark verzögert (u.a., weil immer wieder auf die Antworten des Kunden gewartet werden muss).
        \item Beim Verkaufsprozess eines Automobilhändlers werden häufig Ressourcen verschwendet, da mehrere Prozessschritte von den Mitarbeitern bereits ausgeführt werden (Abschlussinspektion durchführen, Fahrzeug reinigen, Anmeldung vorbereiten), bevor das Ergebnis der Kreditwürdigkeitsprüfung vorliegt.
        \item Bei der Störungsbearbeitung eines Telekommunikationsunternehmens kommt es zu längeren Bearbeitungszeiten, weil die Mitarbeiter während der Ausführung einer Prozessinstanz immer wieder wechseln und die neuen Mitarbeiter sich erneut in den Gegenstand der Störung einarbeiten müssen.
        \item Bei der regelmäßigen Bestellung von Rohstoffen bei einem Lieferanten wird die vom Lieferanten erhaltene Rechnung per Hand in das Buchhaltungssystem übernommen. Neben der dafür erforderlichen Zeit kommt es auch häufig zu Fehler bei der Übernahme der Rechnungsdaten.
    \end{enumerate}

\paragraph*{Lösung}
    \begin{enumerate}[label=\alph*)]
        \item 
        \item 
    \end{enumerate}

\subsubsection*{Aufgabe (Analyse von Prozessen)}
\paragraph*{Aufgabe}
    Sie stellen (subjektiv) fest, dass Sie häufig bestimmte Lebensmittel nicht im Haus haben, die Sie eigentlich benötigen. Analysieren Sie Ihren (einen hypothetischen) Einkaufsprozess mit dem Ursachen-Wirkungs-Diagramm auf mögliche Ursachen für diese Beobachtung.
\paragraph*{Lösung}


\subsubsection*{Aufgabe (Systembegriff)}
\paragraph*{Aufgabe}
    Versetzen Sie sich in folgende Situation: Sie möchten den Einkaufsprozess in Ihrem familiären Umfeld bzw. in Ihrer Haushaltsgemeinschaft durch die Nutzung einer Einkaufslisten-App verbessern. Das SOLL-Modell zum Prozess ist bereits entwickelt worden:
   
    \begin{itemize}
        \item Schritt 1: Haushaltsmitglieder tragen Ihre Einkaufswünsche in die App ein.
        \item Schritt 2 (immer Mittwochs): Die App erstellt aus den Einkaufswünschen der Haushaltsmitglieder eine gesammelte Einkaufsliste und sortiert diese gemäß des Aufbaus des ausgewählten Lebensmittelgeschäfts)
        \item Schritt 3: Die mit dem Einkaufen beauftragte Personen fährt zum Lebensmittelgeschäft und führt den Einkauf gemäß der Einkaufsliste durch.
        \item Schritt 4: Die eingekauften Produkte werden in die dafür vorgesehenen Schränke eingeräumt.
    \end{itemize}
\paragraph*{Lösung}

