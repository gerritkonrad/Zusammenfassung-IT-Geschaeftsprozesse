\section{Betriebliche Anwendungssysteme}

\subsection{Definition und Klassifikation betrieblicher Anwendungen}
    \subsubsection*{Definition}
        Betriebliche Anwendungssysteme sind Softwarelösungen, die zur Unterstützung und Optimierung von Geschäftsprozessen in Unternehmen dienen.
    \subsubsection*{Klassifikation}
        Sie werden in verschiedene Kategorien eingeteilt, wie z.B. Transaktionssysteme, Planungssysteme, und Kontrollsysteme.
    \subsubsection*{Integration}
        Horizontale Integration (Funktionsbereichsübergreifend) und vertikale Integration (überunterschiedliche Managementebenen hinweg) sind wesentliche Aspekte.

\subsection{ERP-Systeme (Enterprise-Resource-Planning)}
    \subsubsection*{Funktion}
        ERP-Systeme sind integrierte Anwendungssysteme, die zur Planung und Steuerung der unternehmensweiten Ressourcen eingesetzt werden. Zu den verwalteten Ressourcen gehören Materialien, Personal, Finanzmittel und mehr.
    \subsubsection*{Integration}
        ERP-Systeme bieten sowohl horizontale als auch vertikale Integration. Horizontale Integration unterstützt operative Geschäftsprozesse in verschiedenen Funktionsbereichen. Vertikale Integration ermöglicht analytische Funktionen zur Berichterstellung und Entscheidungsunterstützung.
    \subsubsection*{End-to-End-Prozesse}
        ERP-Systeme unterstützen durchgehende Prozesse wie den Order-to-Cash und Procure-to-Pay-Prozess ohne Medienbrüche.

\subsection{Standard- vs. Individualsoftware}
    \subsubsection*{Standardsoftware}
        Software, die von einem Hersteller für den Einsatz in vielen verschiedenen Unternehmen entwickelt wird. Sie bietet eine breite Funktionalität und hohe Qualität, erfordert jedoch Anpassungen für spezifische Geschäftsprozesse.
        \begin{itemize}
            \item Vorteile: Höhere Qualität, umfangreiche Funktionen, kontinuierliche Weiterentwicklung.
            \item Nachteile: Geringe Abbildung individueller Prozesse, Abhängigkeit vom Hersteller, hoher Einführungsaufwand.
        \end{itemize}
    \subsubsection*{Individualsoftware}
        Speziell für die Bedürfnisse eines einzelnen Unternehmens entwickelt. Sie bietet maßgeschneiderte Lösungen, kann aber teurer und aufwändiger in der Entwicklung sein.
    \subsubsection*{Anpassungen}
        Standardsoftware kann durch Customizing, Erweiterungsprogrammierung und Modifikation an spezifische Anforderungen angepasst werden. Release Fähigkeit: Alte individuelle Anpassungen sind nach dem Einspielen des nächsten Releases automatisch wieder verfügbar

\subsection{Einführung von ERP-Systemen}
    \subsubsection*{Chancen}
        Verbesserte Prozessstandardisierung, zentrale Datenspeicherung und Integration von Unternehmensprozessen.
    \subsubsection*{Risiken}
        Fehlende Nutzung aller Funktionen, Unterschätzung des Einführungsaufwands, Schwierigkeiten bei der Datenmigration und hohen Kosten durch Softwareanpassungen.
    \subsubsection*{Dienstleister}
        Oft übernehmen externe Dienstleister den Betrieb der ERP-Systeme, wobei die Leistungen in Service-Level-Agreements festgehalten sind.
    \subsubsection*{Vorgehensmodell}
        \begin{itemize}
            \item Initialisierung \& Situationsanalyse
            \item Entwicklung des Sollkonzepts
            \item Marktanalyse
            \item Systemauswahl
            \item Realisierung
            \item Einführung \& Betrieb
        \end{itemize}

\subsection{Anwendungslandschaften}
    \subsubsection*{Definition}
        Die Gesamtheit der in einer Organisation betriebenen betrieblichen Anwendungen und deren Verbindungen.
    \subsubsection*{Entwicklung}
        Früher monolithische Systeme mit Medienbrüchen, heute service-orientierte Architekturen mit standardisierten Schnittstellen für eine nahtlose Integration.
    \subsubsection*{Integration}
        Moderne Anwendungslandschaften ermöglichen eine softwaretechnische Integration ohne Medienbrüche.