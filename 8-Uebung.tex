\subsection{Übung}

\subsubsection*{Aufgabe 10.1}
    \paragraph*{Aufgabe}
        Nennen und beschreiben Sie die in der Vorlesung besprochenen Markttransaktionsphasen.
    \paragraph*{Lösung}
        \begin{itemize}
            \item Informationsphase: Beschaffung von Information
            \item Vereinbarungsphase: Aushandlung der Zahlungs- und Lieferkonditionen. Übermittlung der erforderlichen Daten, Abschluss des Kaufvertrags.
            \item Abwicklungsphase: Austausch der vereinbarten Leistungen
        \end{itemize}

\subsubsection*{Aufgabe 10.2}
    \paragraph*{Aufgabe}
        Ordnen Sie die nachstehenden Funktionen den Markttransaktionsphasen zu:
        \begin{enumerate}[label=\alph*)]
            \item Kunden können die von Ihnen konfigurierten Möbel in einer Simulation betrachten.
            \item Ein Kunde kann zur Bezahlung einer Bestellung einen elektronischen Zahlungsdienst
            einsetzen.
            \item Eine gekaufte Software kann alternativ zum Versandt einer CD auch direkt nach dem Kauf heruntergeladen werden.
            \item Bei dem Abschluss einer Bestellung kann ein Kunde einen Rabattcode angeben, der unmittelbar zu einem Nachlass auf den Rechnungsbetrag führt.
        \end{enumerate}
    \paragraph*{Lösung}
        \begin{enumerate}[label=\alph*)]
            \item Informationsphase
            \item Vereinbarungsphase
            \item Abwicklungsphase
            \item Vereinbarungsphase
        \end{enumerate}

\subsubsection*{Aufgabe 10.3}
    \paragraph*{Aufgabe}
        \begin{enumerate}[label=\alph*)]
            \item Wofür stehen die Akronyme B2B und B2C?
            \item Was ist der Unterschied zwischen den mit B2B und B2C bezeichneten Geschäftstransaktionen?
            \item Geben Sie je ein Beispiel für eine B2B- und eine B2C-Geschäftstransaktion an.
        \end{enumerate}
    \paragraph*{Lösung}
        \begin{enumerate}[label=\alph*)]
            \item B2B = Business to Business, B2C = Business to Customer
            \item B2B = Kontakt mit anderem Unternehmen (Zwischenbetriebliche Interaktion). B2C = Kontakt mit Endkunden.
            \item B2B = Verkauf von Fahrzeugen vom Hersteller an Autohaus.
            \item B2C = Verakuf von Fahrzugen vom Autohaus an Endkunden.
        \end{enumerate}

\subsubsection*{Aufgabe 10.4}
    \paragraph*{Aufgabe}
        Worin besteht die Aufgabe eines zwischenbetrieblichen Informationssystems
    \paragraph*{Lösung}
        Unterstützung der elektronischen Abwicklung von Geschäftsprozessen zwischen verschiedenen Unternehmen und der öffentlichen Verwaltung

\subsubsection*{Aufgabe 10.5}
    \paragraph*{Aufgabe}
        \begin{enumerate}[label=\alph*)]
            \item Warum ist die Verwendung von Standards zu Gestaltung von zwischenbetrieblichen Informationssystemen sinnvoll?
            \item Was versteht man unter dem EDIFACT-Standard?
        \end{enumerate}
    \paragraph*{Lösung}
        \begin{enumerate}[label=\alph*)]
            \item So können alle Unternehmen problemlos und einheitlich miteinander Kommunizieren.
            \item Electronic Data Interchange for Administration, Commerce and Transport.
            Ein Standard zur Strukturierung von Dokumenten und Austausch von Informationen.      
        \end{enumerate}
