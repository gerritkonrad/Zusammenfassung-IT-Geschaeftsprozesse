\section{Datenmodellierung}

\subsection{Ziel der ER-Modelle}
    \begin{itemize}
        \item Darstellung der im betrieblichen Informationssystem relevanten Daten.
        \item Visualisierung der Datenobjekte (Entitäten) mit ihren Attributen und den Beziehungen zwischen ihnen auf Typebene.
    \end{itemize}

\subsection{Modellelemente von ER-Modellen}
    \subsubsection*{Entitäten}
        \begin{itemize}
            \item Identifizierbare und abgrenzbare Datenobjekte (z.B. Kunde, Rechnung)
            \item Besitzen Attribute und Beziehungen zu andren Entitäten
            \item Darstellung durch Rechtecke
        \end{itemize}
    \subsubsection*{Attribute}
        \begin{itemize}
            \item Relevante Eigenschaften einer Entität (z.B. Name, Rechnungsbetrag)
            \item Schlüsselattribute (identifizieren die Entitäten eindeutig) werden unterstrichen
            \item Darstellung durch Ovale
        \end{itemize}
    \subsubsection*{Relationen}
        \begin{itemize}
            \item Beziehungen zwischen Entitäten (z.B. Kunde <-> Rechnung)
            \item Darstellung durch Rauten, die mit den in Beziehungen stehenden Entitäten verbunden werden
        \end{itemize}

\subsection{Kardinalitäsverhältnisse}
    \begin{itemize}
        \item Beschreiben die Anzahl der Entitäten, die an einer Beziehung teilnehmen können.
        \item 1:1-Beziehung: Eine Entität ist mit genau einer anderen Entität verbunden
        \item 1:n-Beziehung: Eine Entität der ersten Art mit mehreren Entitäten der zweiten Art verbunden, aber jede Entität der zweiten Art nur mit einer der ersten Art
        \item n:n-Beziehung: Entitäten beider Art können beliebig oft miteinander in Beziehung stehen
    \end{itemize}