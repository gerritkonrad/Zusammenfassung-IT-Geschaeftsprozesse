\section{E-Commerce}

\subsection{Grundlagen eCommerce}
    \subsubsection*{Definition}
        Abwicklung von Markttransaktionen unter Verwendung von Internet-Technologien
    \subsubsection*{Ziel}
        Prozesse mit externen Akteuren effektiver und effizienter zu gestalten
    \subsubsection*{Markttransaktionsphasen}
        \begin{itemize}
            \item Informationsphase: Beschaffung von Informationen (Suchkosten)
            \item Vereinbarungsphase: Aushandlung der Zahlungs- und Lieferkonditionen, Übermittlung der erforderlichen Daten, Abschluss des Kaufvertrags.
            \item Abwicklungsphase: Austausch der vereinbarten Leistung (Transportkosten, Überweisungskosten)
        \end{itemize}
    \subsubsection*{Transaktionskosten}
        Kosten, die in den verschiedenen Phasen der Ausführung einer Markttransaktion anfallen.
    \subsubsection*{Unterscheidung nach Geschäftspartner} 
        \begin{itemize}
            \item Business 2 Business (B2B): Zwischenbetriebliche Interaktion
            \item Business 2 Customer (B2C): Absatz von Produkten und Dienstleistungen an Endkunden
            \item Business 2 Government (B2G): Geschäftsbeziehung zwischen Unternehmen und der öffentlichen Verwaltung
            \item Customer 2 Customer (C2C): Handel von Produkten zwischen Endkunden
        \end{itemize}

\subsection{Zwischenbetriebliche Informationssysteme (B2B/B2G)}
    \subsubsection*{Definition}
        Unterstützung der elektronischen Abwicklung von Geschäftsprozessen zwischen verschiedenen Unternehmen und zwischen Unternehmen und der öffentlichen Verwaltung.
    \subsubsection*{Aufgabe}
        Austausch von Informationen zwischen Akteuren durch gemeinsame Kommunikationsbeziehungen und Vereinbarungen zur Interpretation der Daten.
    \subsubsection*{Standards}
        EDIFACT (Electronic Data Interchange for Administration, Commerce and Transport) ist ein Internationaler Standard zur Strukturierung von Dokumenten zwischenbetrieblichen Austausch von Informationen. Dieser Umfasst über 200 Nachrichtentypen (Rechnungen, Bestellungen, Produktstammdaten)

\subsection{Außenwirksame Informationssysteme}
    \subsubsection*{Definition}
        Systeme, die sich an externe Anwender richten (Kunden, Geschäftskunden, Lieferanten, Dienstleister, Behörden) und die IT-gestütze Abwicklungen von unternehmensübergreifenden Geschäftsprozessen ermöglichen
    \subsubsection*{Klassifikation}
        \begin{itemize}
            \item Unterstützte Funktionsbereiche
            \item Unterstützte Prozessebene
            \item Produkt- und Branchenorientierung
            \item Unterstütze Markttransaktionsphasen
            \item Adressierte Zielgruppen
            \item Konzeptionelle Ausrichtung
            \item IS-Betreiber
        \end{itemize}

\subsection{Spezielle außenwirksame Informationssysteme}
    \subsubsection*{Unternehmensinformationsportale}
        \begin{itemize}
            \item Bieten externen Benutzern (Kunden, Lieferanten) Zugriff auf Informationen über das Unternehmen und Produkte
            \item Bieten Informationen eines internen Informationssystems für angemeldete Benutzer (z.B. Lagerbestand)
        \end{itemize}
    \subsubsection*{eShops}
        Darstellung/Verkauf von Produkten/Dienstleistungen \\
        Unterstützung der Transaktionsphasen:
        \begin{itemize}
            \item Informationsphase: Detaillierte Produktdarstellung.
            \item Vereinbarungsphase: Bestimmung Zahlungs- und Lieferbedingungen.
            \item Abwicklungsphase: Zahlungsabwicklung.
        \end{itemize}
    \subsubsection*{Lieferantenprotale}
        \begin{itemize}
            \item Integration von Lieferanten in die Informationsverarbeitung.
            \item Unterstützung des Einkaufprozesses durch Bereitstellung von Produktkatalogen und Abwicklung der Bestellungen.
        \end{itemize}

\subsection{Internetdienste im Zusammenhang des E-Commerce}
    \subsubsection*{Suchdienste}
        Ermöglichen die Suche nach bestimmten Inhalten (Webseiten, Produkten, Dienstleistungen) im Internet
    \subsubsection*{Klassifikationen}
        \begin{itemize}
            \item Gegenstand der Suche: Objekte, Personen, Produkte, Dienstleistungen
            \item Bereich der Suche: Internetdienste, begrenzter Raum (unternehmen, regional), medienbezogen (Text, Ton, Bild, Video)
            \item Verfahren der Suche: Indexbasierte Stichwortsuche, indexbasierte Volltextsuche, semantische Suche
        \end{itemize}