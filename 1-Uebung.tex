\subsection{Übung}

\subsubsection*{Aufgabe (Systembegriff)}
\paragraph*{Aufgabe}
    Bestimmen Sie für die nachstehenden Aussagen, ob diese wahr oder falsch sind. Geben Sie für die als „falsch“ identifizierten Aussagen eine Korrektur an:
    \begin{enumerate}[label=\alph*)]
        \item Die Systemstruktur eines Systems legt fest, wie die Komponenten eines Systems miteinander interagieren.
        \item Als atomar werden Komponenten innerhalb eines Systems bezeichnet, die nicht weiter zerlegt sind.
        \item Die Komponenten eines Systems sind die Elemente des Systems, die nicht miteinander interagieren können.
        \item Die Schnittstelle einer Komponente legt fest, wie mit der Komponente interagiert werden kann.
    \end{enumerate}
\paragraph*{Lösung}
    \begin{enumerate}[label=\alph*)]
        \item Falsch, das Systemverhalten legt fest, wie die Komponenten miteinander kommunizieren können.
        \item Richtig
        \item Falsch, die Komponenten können miteinander kommunizieren
        \item Wahr
    \end{enumerate}

\subsubsection*{Aufgabe (Persönliches Informationssystem)}
\paragraph*{Aufgabe}
    Nennen Sie zwei Beispiele, wie Sie durch Ihr persönliches Informationssystem in alltäglichen Tätigkeiten unterstützt werden. Beschreiben Sie dabei, inwiefern das Informationssystem Daten erzeugt, speichert, ausgibt, überträgt und/oder verarbeitet. \\
    Überlegen Sie sich zwei Faktoren, die die kontinuierliche Veränderung Ihres persönlichen Informationssystems beeinflussen und geben Sie jeweils ein Beispiel für die genannten Faktoren an.
\paragraph*{Lösung}
    Beispiel 1: Notizen auf dem Handy, kann von mir angeguckt und ausgelesen werden \\
    Beispiel 2: Smart Home. Eingabe in Handy oder Sprache. Speicherung auf Gerät. Ausgabe: Licht geht an oder eine Szene spielt ab. Übertragung findet über WLAN statt. 
    
    Erweiterung an vorhandene Systeme z.B. kauf eines smarten Kühlschranks, der ins vorhande Smart Home eingebunden wird. \\
    Änderung von System (Architektur). Systeme entwickeln sich weiter bzw. wechselt man eventuell auf einen anderen Hersteller.

\subsubsection*{Aufgabe (Betriebliche Informationssysteme)}
\paragraph*{Aufgabe}
    Was ist ein Sozio-Technisches System und warum werden betriebliche Informationssysteme als sozio-technische Systeme bezeichnet. Verdeutlichen Sie ihre Antwort an einem konkreten Beispiel.
\paragraph*{Lösung}
    Ein System, wo Menschen mit einem Computer interagieren müssen, um die Aufgabe zu erfüllen. Bsp. Kassensystem \textrightarrow Mensch tippt Preise ein, System verarbeitet Preise

\subsubsection*{Aufgabe (Einflussfaktoren betrieblicher Informationssysteme)}
\paragraph*{Aufgabe}
    Inwiefern hat die Einführung der Bong-Pflicht im Jahr 2020 zu einer Anpassung der Informationssysteme vieler (kleinerer) Unternehmen geführt.
\paragraph*{Lösung}
    Da Sie gezwungen sind jeden Einkauf genau zu dokumentieren, mussten neue Systeme angeschafft bzw. alte Systeme umgebaut werden.

\subsubsection*{Aufgabe (Technologien und Konzepte zur Unterstützung von Geschäftsprozessen)}
\paragraph*{Aufgabe}
    Beschreiben Sie für die nachstehend genannten Anwendungsfälle, wie diese durch eine in der Vorlesung vorgestellten Technologien bzw. Konzepte (RFID, Internet der Dinge, virtuelle bzw. erweiterte Realität) sinnvoll unterstützt werden (können).
    \begin{itemize}
        \item Bestellung von Druckertinte
        \item Kauf einer neuen Küche
        \item Rückgabe von ausgeliehenen Werkzeugen (bei einem professionellenWerkzeugverleih)
    \end{itemize}
\paragraph*{Lösung}
    \begin{itemize}
        \item Internet der Dinge \textrightarrow Drucker bestellt Tinte selbst
        \item Virtuelle Realität \textrightarrow Man kann sie die neue Küche genau ansehen, da sie virtuell in das bestehende Zimmer gesetzt wird.
        \item RFID \textrightarrow Chip wird gescannt, und das System erkennt, dass das Gerät zurück gegeben wurde
    \end{itemize}

\subsubsection*{Aufgabe (Entscheidungsunterstützung durch Informationsverarbeitung)}
\paragraph*{Aufgabe}
    Sie möchte Ihre Einkaufskosten zur Beschaffung der privaten Lebensmittel reduzieren. Nehmen Sie an, dass Sie die in den vergangenen Jahren Ihre Einkaufslisten konsequent digital gepflegt haben und nun über einen sehr großen Datensatz an Einkaufspositionen verfügen. Der Datensatz umfasst (mindestens) Angaben zu: \\
    Gekauftes Lebensmittel, Einkaufspreis, Geschäft (in dem gekauft wurde) \\
    Wie könnten Sie diese Informationen verwenden, um Ihre Einkaufskosten ggf. zukünftig zu reduzieren?
\paragraph*{Lösung}
    Vergleich der Preise von gleichen Produkten in verschiedenen Läden. So kann man den günstigsten Preis finden.



