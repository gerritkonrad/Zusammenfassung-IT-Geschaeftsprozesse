\section{Einführung}
\subsection{Betriebliche Informationssysteme}
\begin{itemize}
    \item Unterstützen und koordinieren operative und strategische Geschäftsprozesse. 
    \item Beispiele sind Kundenauftragsprozesse, Produktionsaufträge und Transportaufträge. 
    \item Transaktionsinformationen werden während der Prozesse erfasst, verarbeitet und ausgegeben.
\end{itemize}

\subsection{Managementprozesse}
\begin{itemize}
    \item Informationen werden zur Entscheidungsunterstützung analysiert und verarbeitet.
    \item Beispiele sind Berichte über die Bearbeitungsdauer von Reklamationen und Ablaufplanung von Produktionsaufträgen.
\end{itemize}

\subsection{Daten und Informationen}
\begin{itemize}
    \item Daten sind maschinenlesbare Repräsentationen von Informationen.
    \item Die Nutzung von Daten erfordert Vereinbarungen zur Interpretation.
\end{itemize}

\subsection{Systeme und Schnittstellen}
\begin{itemize}
    \item Ein System besteht aus Komponenten, die miteinander interagieren.
    \item Schnittstellen definieren die Interaktionsmöglichkeiten und den Datenaustausch zwischen Komponenten.
    \item Kommunikationsverbindungen sind für den eigentlichen Datenaustausch verantwortlich.
\end{itemize}

\subsection{Informationssysteme}
\begin{itemize}
    \item Erzeugen, speichern, übertragen und verarbeiten Informationen.
    \item Bestehen aus Menschen und Maschinen und sind sozio-technische Systeme.
    \item Rechnergestützte betriebliche Informationssysteme nutzen Informationstechnologie zur Unterstützung von Geschäftsprozessen.
\end{itemize}

\subsection{Einflussfaktoren}
Änderungen in geschäftlichen Anforderungen, gesetzlichen Vorgaben und technischen Innovationen erfordern kontinuierliche Anpassungen der Informationssysteme.

\subsection{Weitere Informationssysteme}
\begin{itemize}
    \item Persönliche Informationssysteme zur Verwaltung persönlicher Daten.
    \item Zwischenbetriebliche Informationssysteme für den Austausch von Informationen mit Geschäftspartnern und Behörden.
    \item Konsumenteninformationssysteme regeln den Informationsaustausch zwischen Unternehmen und Kunden.
\end{itemize}

\subsection{Technologien}
Beispiele wie RFID zur Identifikation und Lokalisierung von Objekten.
