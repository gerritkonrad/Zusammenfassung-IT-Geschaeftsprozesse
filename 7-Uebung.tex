\subsection{Übung}

\subsubsection*{Aufgabe 9.1 (Funktionen von ERP-Systemen)}
    \paragraph*{Aufgabe}
        In einer Hochschule soll ein ERP-System zur Unterstützung der operativen Abläufe eingeführt
        werden.
        \begin{enumerate}[label=\alph*)]
            \item Nennen Sie einerseits die Anwendungskomponenten eines konventionellen ERP-Systems, die sinnvoll in einer Hochschule eingesetzt werden können und andererseits
            die Anwendungskomponenten, die im Anwendungsbereich einer klassischen Hochschule keine Bedeutung haben. Begründen Sie für jede Anwendungskomponente Ihre Entscheidung.
            \item Erläutern Sie mindestens eine Funktionalität aus dem Bereich der Prüfungsverwaltung einer Hochschule, die von einem konventionellen ERP-System nicht unterstützt
            werden wird.
            \item Welche Möglichkeit besteht für eine Hochschule, ein ERP-System zu erwerben, dass typische Funktionen z.B. aus dem Bereich der Prüfungsverwaltung unterstützt?
        \end{enumerate}
    \paragraph*{Lösung}
        \begin{enumerate}[label=\alph*)]
            \item
        \end{enumerate}

\subsubsection*{Aufgabe 9.2 (Anpassung von ERP-Systemen)}
    \paragraph*{Aufgabe}
        \begin{enumerate}[label=\alph*)]
            \item Welche Möglichkeiten existieren zur Anpassung eines (ggf. branchenspezifischen) ERP-Systems an die individuellen Anforderungen eines Unternehmens?
            \item Erläutern Sie den Begriff der Releasfähigkeit.
            \item Durch welche Art der Anpassung wird die Relasfähigkeit am stärksten beeinträchtigt? Begründen Sie Ihre Antwort.
        \end{enumerate}       
    \paragraph*{Lösung}
        \begin{enumerate}[label=\alph*)]
            \item Customising, Erweiterung, Modifikation.
            \item Alte individuelle Anpassungen sind nach dem Einspielen des nächsten Releases automatisch wieder verfügbar.
            \item Modifikation. Funktionen werden komplett neu entwickelt. Es kann in neuen Versionen nicht garantiert werden eine passende Schnittstelle für selbst entwickelte Funktionen zu haben.
        \end{enumerate}

\subsubsection*{Aufgabe 9.3 (Architektur von ERP-Systemen)}
    \paragraph*{Aufgabe}
        Welche Konsequenzen ergeben sich, wenn das Basissystem eines ERP-Systems über keinerlei
        Integrationskomponenten verfügt?
    \paragraph*{Lösung}
        Das System kann nicht mit Fremdsystemen kommunizieren.

\subsubsection*{Aufgabe 9.4 (Standard vs. Individualsoftware)}
    \paragraph*{Aufgabe}
        Beschreiben Sie drei Nachteile, die bei dem Einsatz einer Standardsoftware im Vergleich zur
        Entwicklung einer Individualsoftware entstehen?
    \paragraph*{Lösung}
        

\subsubsection*{Aufgabe 9.5 (Einführung von ERP-Systemen)}
    \paragraph*{Aufgabe}
        \begin{enumerate}[label=\alph*)]
            \item Nennen Sie die sechs Phasen des in der Vorlesung besprochenen Vorgehensmodells zur Einführung eines ERP-Systems.
            \item Welche Aufgabe übernehmen Key-User in der Phase „Systemauswahl“? Inwiefern können die gewonnenen Erkenntnisse der Key-User in die Fit-Gap-Analyse einfließen?
        \end{enumerate}
    \paragraph*{Lösung}
        \begin{enumerate}[label=\alph*)]
            \item 
            \item 
        \end{enumerate}

\subsubsection*{Aufgabe 9.6 (Betrieb von ERP-Systemen)}
    \paragraph*{Aufgabe}
        Ein kleines Gärtnerei-Unternehmen will eine neue ERP-Software einführen. Das Unternehmen verfügt derzeit über keine IT-Abteilung. Welche Optionen bieten sich für das Unternehmen, trotz der fehlenden IT-Abteilung ein ERP-System zu betreiben? Nennen Sie die beiden möglichen Optionen und diskutieren Sie die deren Eignung für das betrachtete Unternehmen?
    \paragraph*{Lösung}
        