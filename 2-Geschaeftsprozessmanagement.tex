\section{Geschäftsprozessmanagement}

\subsection{Grundlagen zu Geschäftsprozessen}
\begin{itemize}
    \item Definition: Geschäftsprozesse sind die arbeitsteilige Ausführung von Aktivitäten in einer zeitlich-/sachlogischen Reihenfolge zur Erfüllung einer betrieblichen Aufgabe.
    \item Sichten: Steuerungssicht, Funktionssicht, Datensicht, Organisationssicht, Leistungssicht.
    \item Arten: Wertschöpfende Kernprozesse, unterstützende Prozesse, Management Prozesse.
    \item Prozesstyp vs. Prozessinstanz: Typ ist die Vorlage, Instanz die konkrete Ausführung.
\end{itemize}

\subsection{Merkmale und Lebenszyklus des Geschäftsprozessmanagements}
\begin{itemize}
    \item Ziele: Prozesse effektiver und effizienter gestalten.
    \item Lebenszyklus: Identifikation, Erhebung, Analyse, Verbesserung, Einführung und Überwachung von Prozessen.
    \item Rollen:
    \begin{itemize}
        \item Geschäftsführung: Verantwortlich für die grundsätzliche Gestaltung der Geschäftsprozesse.
        \item Prozessverantwortlicher: Verantwortlich für die Ausführung und Anpassung der Prozesse.
        \item Prozessteilnehmer: Führen Routineaufgaben innerhalb der Prozesse aus.
        \item Systemanalytiker: Erhebt, analysiert und verbessert Prozesse.
        \item Anwendungsentwickler: Verantwortlich für die softwaretechnische Umsetzung.
    \end{itemize}
\end{itemize}

\subsection{Identifikation von Geschäftsprozessen}
\begin{itemize}
    \item Vorgehen: Erfassung der wichtigsten Prozesse, Darstellung als Prozesslandkarte oder Wertschöpfungskette, Bewertung und Auswahl der zu verbessernden Prozessen.
    \item Referenzmodelle: Dienen als Vorlagen zur Entwicklung spezifischer Prozesse. Bespiel: Handels-H-Modell (siehe unten). Anerkannte Lösung für wiederkehrende Probleme.
    \item Techniken: Prozessmodellierung und -analyse zur Optimierung.
\end{itemize}

\subsection{Gestaltung von Geschäftsprozessen}
    \subsubsection{Prozesse erheben}
        \begin{itemize}
            \item Ziel: Sammeln von Informationen über den aktuellen Ablauf eines Prozesses (IST-Modell).
            \item Methoden:
            \begin{itemize}
                \item Dokumentensichtung: Nutzt vorhandene Dokumente, kann aber veraltet sein.
                \item Beobachtung: Direkte Erkennung des IST-Zustands, jedoch zeitaufwendig.
                \item Interviews: Detaillierte Informationen, aber zeitintensiv.
                \item Workshops: Kompakte und kollaborative Erhebung, jedoch zeitaufwendig für alle.
            \end{itemize}
        \end{itemize}

    \subsubsection{Prozesse analysieren} 
        \begin{itemize}
            \item Ziel: Identifizieren von Schwachstellen und deren Ursachen.
            \item Typische Schwachstellen:
            \begin{itemize}
                \item Lange Durchlaufzeiten
                \item Hohe Fehlerquote
                \item Hohe Kosten
                \item Geringe Flexibilität
            \end{itemize}
            \item Methoden:
            \begin{itemize}
                \item Qualitative Analyse: Wertbeitragsanalyse, Ursache-Wirkungsdiagramme.
                \item Quantitative Analyse: Nutzung statistischer Daten zur Identifikation von Engpässen.
            \end{itemize}
        \end{itemize}

    \subsubsection{Prozesse verbessern}  
        \begin{itemize}
            \item Ziel: Vorschläge zur Eliminierung von Schwachstellen und Erstellung eines SOLL-Prozesses.
            \item Dimensionen der Verbesserung: Durchlaufzeit, Kosten, Qualität, Flexibilität.
            \item Methoden:
            \begin{itemize}
                \item Verbesserungsvorschläge in den genannten Dimensionen erarbeiten.
                \item Redesign-Heuristiken: Konkrete Maßnahmen zur Umgestaltung von Prozessen.
            \end{itemize}
        \end{itemize}

\subsection{Ausführung von Geschäftsprozessen}
\begin{itemize}
    \item Umsetzung: Prozesse in die Praxis umsetzen durch Implementierung und Anpassung von Anwendungssystemen.
    \item Beispiele: Implementierung neuer Systeme, Anpassung bestehender Systeme, Bereitstellung benötigter Geräte.
\end{itemize}

\subsection{Prozesse einführen}
\begin{itemize}
    \item Definition: Organisatorische und technische Maßnahmen zur Bereitstellung der Infrastruktur.
    \item Maßnahmen: Mitarbeiterschulung, Schaffung neuer Stellen, Anpassung bestehender Stellen, Implementierung oder Anpassung von Anwendungssystemen.
\end{itemize}

\subsection{Prozesse überwachen}
\begin{itemize}
    \item Prozessüberwachung: Überwachung anhand aufgezeichneter Daten.
    \item KPI (Key Performance Indicators): Aggregation und Darstellung von Kennzahlen zur Überwachung und Identifikation von Optimierungspotentialen.
\end{itemize}